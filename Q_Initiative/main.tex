\documentclass{beamer}
%%%%%%%%%%%%%%%%%%%%%%%%%%%%%%%  Packages  %%%%%%%%%%%%%
\usepackage{amsmath} 
\usepackage{mathtools}
\usepackage{physics}
\usepackage{amssymb}
\usepackage{mathptmx}
\usepackage{array}
  
\usepackage[sort&compress]{natbib}     % bib

%%%%%%%%% FIGurES %%%%%%%%%%%%%%%%%%%%%%%%
\usepackage{textcomp}
\usepackage{graphicx}
\usepackage{caption} 
\usepackage{subcaption}
\usepackage{scrextend}
\usepackage{rotating}
\usepackage{float}
\usepackage{hyperref}

\usepackage{multimedia}

%\graphicspath{{./figures/}}
\hypersetup{colorlinks=true, citecolor=blue, linkcolor=blue}
 
%%%%%%%%%%%% LaNgUaGe %%%%%%%%%%%%%%%%%%
\usepackage{verbatim}
\usepackage{natbib}
\usepackage{wrapfig}
\usepackage[utf8]{inputenc}

%%%%%%%%%%%%%% PhYsIcS %%%%%%%%%%%%%%%%%%%%%%%

\usepackage{qcircuit}

%%%%%%%%%%%%%%%%%5 TIKZ %%%%%%%%%%%%%%%%%%%%

\usepackage{tikz}
\usetikzlibrary{positioning,calc}
\newcommand{\tikzmark}[3][]{\tikz[remember picture,baseline] \node [anchor=base,#1](#2) {$#3$};}

%%%%%%%%%%%%%% minted %%%%%%%%%%%%%%%%%%%%%%
\usepackage{minted}
\setminted[python]{breaklines}


\usetheme{PaloAlto}

\title{Python quantum programming languages}
\author{John Scott, Oliver Thomas}
\institute{Quantum Engineering CDT \\ University of Bristol}
\date{\today}


\begin{document}

% slide 1
\frame{\titlepage}

% slide 2
\begin{frame}
\frametitle{Overview}
\begin{itemize}
    \item We'll focus on Python based quantum programming libraries
    \item We tried to program the common programs (e.g. Grover's algorithm, Shor's algorithm, etc.)
    \item We tried compiling a simple program for different hardware platforms (i.e. with gate restrictions, etc.)
    \item We've written a programming guide -- it's under an internal review
\end{itemize}
\end{frame}

\begin{frame}[fragile]
  \begin{minted}{python}
    hello
    print('test')
  \end{minted}
\end{frame}

%slide 3
\begin{frame}
\frametitle{Short comparison}
\begin{columns}
\column{0.46\textwidth}
    \begin{block}{What is there}
    \begin{itemize}
        \item Focussed on quantum circuits
        \item Apply gates to specific qubits
        \item Classical control in the same source code
        \item Python syntax is beginner friendly
        \item Simulators are available
        \item Hardware compilers are available
    \end{itemize}
    \end{block}
    %
\column{0.5\textwidth}
    \begin{block}{What is lacking}
        \begin{itemize}
            \item Lack of support for custom unitaries
            \item Compilers are not highly developed
            \item Some languages target specific hardware
            \item Some simulators are cloud based and require accounts
            \item No real quantum programming contructs (e.g. quantum if etc.)
        \end{itemize}
        \end{block}
\end{columns}
\end{frame}
% slide 4
\begin{frame}
\frametitle{What do we mean by nonlinear optics?}
\begin{itemize} 
    \item Roughly processes that conserve energy but do not conserve photon number. 
\end{itemize}
\end{frame}

%slide 5
\begin{frame}
\frametitle{Gaussian Optics}

    \begin{itemize}
    \item Using th
\end{itemize}

\vspace{10pt} 
\begin{itemize}
    \item We 
\end{itemize}
\vspace{-10pt}
\end{frame}

% slide 6
\begin{frame}
    \frametitle{Types of} 
%
\begin{figure}[h]
\begin{align*}
\centering
\vspace{-20pt}
    \Qcircuit @C=0.6cm @R=0.1cm{
    %1
        &\lstick{a_1} &\qw &\multigate{1}{Squeezer} &\qw &\qw &\measureD{} \\
    %2
        &\lstick{a_2} &\qw &\ghost{Squeezer} &\qw  &\multigate{1}{BS} &\measureD{} \\
    %3
        &\lstick{a_3} &\qw &\multigate{1}{Squeezer} &\qw &\ghost{BS} &\measureD{} \\
    %4
        &\lstick{a_4} &\qw &\ghost{Squeezer} &\qw &\qw &\measureD{} \\
}
\end{align*}
\caption{Two source HOM dip}
\end{figure}
%
    \vspace{-20pt}
% 
\footnotetext{These are two-mode squeezers}
%
\end{frame}


% slide 7 
\begin{frame}{Schmidt decomposition}
    \begin{itemize}
        \item with $ \psi_k(\omega_1) $ is the k-th row and $\omega_1$-th column of $\textbf{U}_{(\omega_1, k)}$,
        \item with $ \phi_k(\omega_2) $ is the $\omega_2$-th row and k-th column of $\textbf{V}^\dagger_{(k,\omega_2)}$
    \end{itemize}
\end{frame}

% slide 8
\begin{frame}
\frametitle{Summary}
k
\end{frame}

% slide references
\begin{frame}{References}
\bibliographystyle{unsrt}
\bibliography{refs}
\end{frame}
\end{document}
