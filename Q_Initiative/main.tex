\documentclass{beamer}
%%%%%%%%%%%%%%%%%%%%%%%%%%%%%%%  Packages  %%%%%%%%%%%%%
\usepackage{amsmath} 
\usepackage{mathtools}
\usepackage{physics}
\usepackage{amssymb}
\usepackage{mathptmx}
\usepackage{array}
  
\usepackage[sort&compress]{natbib}     % bib

%%%%%%%%% FIGurES %%%%%%%%%%%%%%%%%%%%%%%%
\usepackage{textcomp}
\usepackage{graphicx}
\usepackage{caption} 
\usepackage{subcaption}
\usepackage{scrextend}
\usepackage{rotating}
\usepackage{float}
\usepackage{hyperref}

\usepackage{multimedia}

%\graphicspath{{./figures/}}
\hypersetup{colorlinks=true, citecolor=blue, linkcolor=blue}
\renewcommand{\equationautorefname}{Eq.}
\renewcommand{\figureautorefname}{Fig.}
 
%%%%%%%%%%%% LaNgUaGe %%%%%%%%%%%%%%%%%%
\usepackage{verbatim}
\usepackage{natbib}
\usepackage{wrapfig}
\usepackage[utf8]{inputenc}

%%%%%%%%%%%%%% PhYsIcS %%%%%%%%%%%%%%%%%%%%%%%

\renewcommand{\annia}{\hat{a}}
\renewcommand{\annib}{\hat{b}}
\renewcommand{\creata}{\hat{a}^\dagger}
\renewcommand{\creatb}{\hat{b}^\dagger}

\renewcommand{\a}{a^ }
\renewcommand{\b}{b^ }
\renewcommand{\adag}{a^\dagger}
\renewcommand{\bdag}{b^\dagger}

\usepackage{qcircuit}

%%%%%%%%%%%%%%%%%5 TIKZ %%%%%%%%%%%%%%%%%%%%

\usepackage{tikz}
\usetikzlibrary{positioning,calc}

%\tikzset{>=stealth}

%\newcommand{\tikzmark}[1]{\tikz[baseline,remember picture] \node[anchor=base, #1] (#1) {};}
%\newcommand{\tikzmark}[3][]{\tikz[overlay,remember picture,baseline] \node [anchor=base,#1](#2) {#3};}
\newcommand{\tikzmark}[3][]{\tikz[remember picture,baseline] \node [anchor=base,#1](#2) {$#3$};}
\usetheme{PaloAlto}

\title{Python quantum programming languages}
\author{John Scott, Oliver Thomas}
\institute{Quantum Engineering CDT \\ University of Bristol}
\date{\today}


\begin{document}

% slide 1
\frame{\titlepage}

% slide 2
\begin{frame}
\frametitle{Overview}
\begin{itemize}
    \item What is it?
    \item Why do we care about it?
    \item What we've been doing
    \item Outlook
\end{itemize}
\end{frame}

%slide 3
\begin{frame}
\frametitle{Motivation quantum nonlinear optics}
\begin{columns}
\column{0.5\textwidth}
    \begin{block}{The good}
    Spontaneous 
    \begin{itemize}
        \item thing 1 
        \item thing 2 
    \end{itemize}
    i don't know
    \begin{itemize}
        \item thing1  
        \item thing 2
    \end{itemize}
    \end{block}
%
\column{0.5\textwidth}
    \begin{block}{The bad}
        Spontaneous 
        \begin{itemize}
            \item i 
            \item d
        \end{itemize}
        A
        \begin{itemize}
            \item o
            \item n
        \end{itemize}
        \end{block}
\end{columns}
\end{frame}
% slide 4
\begin{frame}
\frametitle{What do we mean by nonlinear optics?}
\begin{itemize} 
    \item Roughly processes that conserve energy but do not conserve photon number. 
\end{itemize}
\end{frame}

%slide 5
\begin{frame}
\frametitle{Gaussian Optics}

    \begin{itemize}
    \item Using th
\end{itemize}
        \begin{equation}
        \hat{U} = \exp[-\frac{i}{\hbar}\left( \tikzmark{P}{P} \int d\omega_1 \int d\omega_2 \tikzmark{f}{f(\omega_1,\omega_2)} \tikzmark{aa}{ \creata_s(\omega_1) \creata_i(\omega_2)} + h.c. \right) ]
    \end{equation}

    \begin{tikzpicture}[overlay, remember picture, node distance=1cm]
    
    \node[red] (power) [below =of P]{Power};
    \draw[red,->,thick] (power) to [in=-90,out=45] (P);
    
    \node[blue,xshift=0cm] (jsa) [below =of f]{JSA};
    \draw[blue,->,thick] (jsa) to [in=-90,out=135] (f.south);
    
    \node[purple] (sigidler) [below =of aa]{Signal \& Idler};
    \draw[purple,->,thick] (sigidler) to [in=-90,out=45] (aa.south);
\end{tikzpicture}
\vspace{10pt} 
\begin{itemize}
    \item We 
\end{itemize}
\vspace{-10pt}
\end{frame}

% slide 6
\begin{frame}
    \frametitle{Types of} 
%
\begin{figure}[h]
\begin{align*}
\centering
\vspace{-20pt}
    \Qcircuit @C=0.6cm @R=0.1cm{
    %1
        &\lstick{a_1} &\qw &\multigate{1}{Squeezer} &\qw &\qw &\measureD{} \\
    %2
        &\lstick{a_2} &\qw &\ghost{Squeezer} &\qw  &\multigate{1}{BS} &\measureD{} \\
    %3
        &\lstick{a_3} &\qw &\multigate{1}{Squeezer} &\qw &\ghost{BS} &\measureD{} \\
    %4
        &\lstick{a_4} &\qw &\ghost{Squeezer} &\qw &\qw &\measureD{} \\
}
\end{align*}
\caption{Two source HOM dip}
\end{figure}
%
    \vspace{-20pt}
% 
\footnotetext{These are two-mode squeezers}
%
\end{frame}


% slide 7 
\begin{frame}{Schmidt decomposition}
    \begin{itemize}
        \item with $ \psi_k(\omega_1) $ is the k-th row and $\omega_1$-th column of $\textbf{U}_{(\omega_1, k)}$,
        \item with $ \phi_k(\omega_2) $ is the $\omega_2$-th row and k-th column of $\textbf{V}^\dagger_{(k,\omega_2)}$
    \end{itemize}
\end{frame}

% slide 8
\begin{frame}
\frametitle{Summary}
k
\end{frame}

% slide references
\begin{frame}{References}
\bibliographystyle{unsrt}
\bibliography{refs}
\end{frame}
\end{document}
